\documentclass{article}

\usepackage[utf8]{inputenc}

\title{Trabalho de MPS}
\author{Victor Cotrim de Lima }

\date{Setembro de 2014}

\begin{document}
\maketitle


\section{Introdução}
O processo definido para este trabalho é o de gerência de decisões, sendo o mesmo baseado em seus objetivos e atividades conforme guia de implementação parte 5 do MPS.BR.\\
O objetivo do processo escolhido é a análise de decisões de cunho crítico conforme critérios pré estabelecidos para diversas alternativas identificadas. Este processo pode ser aplicado a qualquer momento sempre que surgir a identificação de tomadas de decisão de uma questão crítica. O principal motivo do uso deste processo é o seu poder de redução da subjetividade da escolha da decisão, gerando assim uma maior probabilidade de escolha da melhor decisão. (\textbf{colocar referência ao guia de implementação})
\paragraph{Falar sobre como o documento está estruturado}

\section{Processo e atividades} 

	\subsection{Atividade do processo 1}
		Breve explanação de que a ativiade proposta para este processo faz
		
		\subsubsection{Atividades}
			Explicação da atividade como um todo com maior detalhamento
		\subsubsection{Fluxogramas}
			Mostrar como é a interação desta atividade no fluxograma sempre mostrando o fluxograma
		\subsubsection{Papeis}
			Quais papeis estão envolvidos nesta atividade(papeis necessários e detalhes)

	\subsection{Atividade do processo 2}
		Breve explanação de que a ativiade proposta para este processo faz
		
		\subsubsection{Atividades}
			Explicação da atividade como um todo com maior detalhamento
		\subsubsection{Fluxogramas}
			Mostrar como é a interação desta atividade no fluxograma sempre mostrando o fluxograma
		\subsubsection{Papeis}
			Quais papeis estão envolvidos nesta atividade(papeis necessários e detalhes)

	\subsection{Iteração com outros processos}		
		Explicar e mostrar se há interações com outros processos, caso haja, quais são estes processos e como acontece esta interação
	
	\subsection{Possíveis ferramentas e métodos}
		Mostrar e explicar se há possibilidade da utilização quais ferramentas e métodos poderiam ser utilizados para auxiliar o processo


	
\section{Considerações finais}

\section{Referência bibliográficas}



\end{document}
