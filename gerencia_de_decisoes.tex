\documentclass{article}

\usepackage[utf8]{inputenc}

\title{Trabalho de MPS}
\author{Victor Cotrim de Lima }

\date{Setembro de 2014}

\begin{document}
\maketitle


\section{Introdução}
O processo definido para este trabalho é o de gerência de decisões, sendo o mesmo baseado em seus objetivos e atividades conforme guia de implementação parte 5 do MPS.BR.\\
O objetivo do processo escolhido é a análise de decisões de cunho crítico conforme critérios pré estabelecidos para diversas alternativas identificadas. Este processo pode ser aplicado a qualquer momento sempre que surgir a identificação de tomadas de decisão de uma questão crítica. O principal motivo do uso deste processo é o seu poder de redução da subjetividade da escolha da decisão, gerando assim uma maior probabilidade de escolha da melhor decisão. (\textbf{colocar referência ao guia de implementação})
\paragraph{Falar sobre como o documento está estruturado}

\section{Processo e atividades} 


	Resultados esperados:
	GDE1 - Guias organizacionais para a gerência de decisões são estabelecidos e mantidos
	
		Resultado esperado feito basicamente para planejamento. Tudo que falar sobre planejamento(cronograma,orçamento,pareceiros,ferramentas a serem usadas,recursos) estará relacionado a isto.
		Possiveis atividades:
			Planejamento de guias organizacionais
										
	"Guias organizacionais devem, então, ser estabelecidos e mantidos contento descrições dos critérios para início obrigatório do processo Gerência de Decisões"[Guia de implementação parte 5, GDE1]
				
	"O processo formal de decisão pode estar associado à execução de qualquer outro processo, sem haver, contudo, uma relação direta entre eles" [Guia de implementação parte 5, GDE1]
	

	GDE2 - O problema ou questão a ser objeto de um processo formal de tomada de decisão é definido
		
		Possiveis atividades:
			Definir problemas
			
	"... definir um problema erroneamente pode conduzir a um caminho que não levará à solução do problema real. Esta atividade visa assegurar que se pretende resolver o problema correto"[Guia de implementação parte 5, GDE2]

	GDE3 - Critérios para avaliação das alternativas de solução são estabelecidos e mantidos em ordem de importância, de forma que os critérios mais importantes exerçam mais influência na avaliação
		
		Possíveis atividades:
			Definir critérios para avaliação de solução
			
		"A priorização ou a ponderação dos critérios poderá ser feita por uma ou mais pessoas. É interessante que se registre o resultado do trabalho com os motivos que levaram à escolha dos critérios e sua priorização e/ou ponderação."[Guia de implementação parte 5, GDE3]
		
		Para garantir a objetividade dos critérios escolhidos os mesmos não devem ser tendenciosos e devem colaborar para se alcançar o objetivo daquele critério, sendo que, o de maior grau de prioridade possui maior influência.[Parafraseando o guia de implementação]

	GDE4 - Alternativas de solução aceitáveis para o problema ou questão são identificadas
		
				
		
	\subsection{Atividade do processo 1}
		Breve explanação de que a ativiade proposta para este processo faz
		
		\subsubsection{Atividades}
			Explicação da atividade como um todo com maior detalhamento
		\subsubsection{Fluxogramas}
			Mostrar como é a interação desta atividade no fluxograma sempre mostrando o fluxograma
		\subsubsection{Papeis}
			Quais papeis estão envolvidos nesta atividade(papeis necessários e detalhes)

	\subsection{Atividade do processo 2}
		Breve explanação de que a ativiade proposta para este processo faz
		
		\subsubsection{Atividades}
			Explicação da atividade como um todo com maior detalhamento
		\subsubsection{Fluxogramas}
			Mostrar como é a interação desta atividade no fluxograma sempre mostrando o fluxograma
		\subsubsection{Papeis}
			Quais papeis estão envolvidos nesta atividade(papeis necessários e detalhes)

	\subsection{Iteração com outros processos}		
		Explicar e mostrar se há interações com outros processos, caso haja, quais são estes processos e como acontece esta interação
	
	\subsection{Possíveis ferramentas e métodos}
		Mostrar e explicar se há possibilidade da utilização quais ferramentas e métodos poderiam ser utilizados para auxiliar o processo


	
\section{Considerações finais}

\section{Referência bibliográficas}



\end{document}
